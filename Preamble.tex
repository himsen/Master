% standard engelsk opsaetning
\usepackage[utf8]{inputenc}   % ʯÂ
\usepackage[english]{babel}            % engelsk opsÊtning
\usepackage{memhfixc}
\usepackage[T1]{fontenc}
\usepackage{lmodern} % lidt bedre CM font
\usepackage{Palatino}
%\usepackage{kpfonts}
\usepackage{pifont}
\usepackage{amssymb}
\newcommand{\cmark}{\ding{51}}%
\newcommand{\xmark}{\ding{55}}%
\usepackage{lipsum}
%\usepackage{showkeys}

% Maple packages
%\usepackage{maple2e}
%\usepackage{mapleenv}
%\usepackage{mapleplots}
%\usepackage{maplestd2e}
%\usepackage{maplestyle}
%\usepackage{mapletab}
%\usepackage{mapleutil}

% Matematiske symboler og fede tegn i ligninger
\usepackage{amsmath, amssymb, bm, bbm, dsfont, mathtools, mathdots, mathrsfs}
\usepackage{xspace}


% Andet
\usepackage{fixltx2e} % retter noget
\usepackage{hyperref} % g¯r at alle krydsreferencer, citeringer og indholdsfortegnelser bliver lavet om til interne hyperlinks.
\usepackage{afterpage}

% Henvisninger
\usepackage{url}
\usepackage{natbib} 
\citestyle{plain} % Plain Style citations
\makeatletter % Suppress various auxiliary commands in bib-file
\newcommand\Firstpublished[1]{\expandafter\ignorespaces\@gobble}
\newcommand\Editedby[1]{\expandafter\ignorespaces\@gobble}
\newcommand\biband{\expandafter\ignorespaces\@gobble}
\newcommand\Bookreview{\expandafter\ignorespaces\@gobble}
\newcommand\Moviereview{\expandafter\ignorespaces\@gobble}
\makeatother

% Tabeller og s¯jler
\usepackage{array, booktabs, dcolumn}
\newcolumntype{d}[1]{D{,}{,}{#1}} % Justering under komma

% Udvidelse til Matric Milj¯
\makeatletter
\renewcommand*\env@matrix[1][*\c@MaxMatrixCols c]{%
  \hskip -\arraycolsep
  \let\@ifnextchar\new@ifnextchar
  \array{#1}}
\makeatother

% Figurer og farver
\usepackage{graphicx, caption, subfig, xcolor}
\captionsetup{font=small,labelfont=bf}

% Saetninger
\usepackage{amsthm}
%\usepackage[standard,amsmath,amsthm]{ntheorem}

\theoremstyle{plain}
\newtheorem{thm}{Theorem}[section]
\newtheorem*{thm*}{Theorem} % uden nummer
\newtheorem{fthm}{Main Theorem}[section]
\newtheorem*{fthm*}{Main Theorem} % uden nummer
\newtheorem{lem}[thm]{Lemma}
\newtheorem*{lem*}{Lemma}
\newtheorem{prop}[thm]{Proposition}
\newtheorem*{prop*}{Proposition}
\newtheorem{cor}[thm]{Corollary}
\newtheorem*{cor*}{Corollary}
\newtheorem{con}[thm]{Conjecture}

% Definitioner
\theoremstyle{definition}
\newtheorem{defn}[thm]{Definition}
\newtheorem*{defn*}{Definition}

% Bemaerkninger og deslige
\theoremstyle{remark}
\newtheorem{ex}[thm]{Example}
\newtheorem*{ex*}{Example}
\newtheorem{rem}[thm]{Remark}
\newtheorem*{rem*}{Remark}

% Pagestyle simpel dokument
\makepagestyle{mysimplestyle} %Min pagestyle med sidetal
\copypagestyle{mysimplestyle}{empty}
\makeoddhead{mysimplestyle}{Kenneth Rasmussen}{\quad}{\today}
\makeheadrule{mysimplestyle}{\textwidth}{\normalrulethickness}
\makeoddfoot{mysimplestyle}{}{\thepage\ af \thelastpage}{} %KrÊver to oversÊttelser
%\pagestyle{mysimplestyle} % Aktiver Simple pagestyle

% Pagestyle st¯rre opgave
\makeevenfoot{companion}{}{\thepage}{}
\makeoddfoot{companion}{}{\thepage}{}
\makeevenfoot{ruled}{}{\thepage}{}
\makeoddfoot{ruled}{}{\thepage}{}
\makeevenfoot{plain}{}{\thepage}{}
\makeoddfoot{plain}{}{\thepage}{}
\pagestyle{ruled}

% lidt marginer
\setlrmarginsandblock{3cm}{*}{1}
\setulmarginsandblock{3cm}{*}{1}
\setheadfoot{2cm}{\footskip}          % mere h¯jde til header
\checkandfixthelayout[nearest]
\renewcommand{\baselinestretch}{1.3}

% Nummering-/inholdsfortegnelsesdybde
\setsecnumdepth{subsection}
\settocdepth{subsection}

% Punktopstilling
\usepackage{enumerate}
\renewcommand{\labelitemi}{$\textbf{-}$}

% Til specielle matricer
\usepackage{blkarray, gauss}

%\R for reelle tal og \C for komplekse og \N og de normale
\newcommand{\R}{\mathbb{R}}
\newcommand{\C}{\mathbb{C}}
\newcommand{\N}{\mathbb{N}}
\newcommand{\Z}{\mathbb{Z}}
\newcommand{\Q}{\mathbb{Q}}
\newcommand{\K}{\mathbb{K}}
\renewcommand{\P}{\mathbb{P}}
%\newcommand{\D}{\mathbb{D}}
\newcommand{\F}{\mathbb{F}}
\newcommand{\infi}{\mathcal{O}} %point at infinity

% Kommandoer specielt til p-adiske tal
\newcommand{\p}[1]{\left\lvert #1 \right\rvert _p} % p-adisk absolut vÊrdi
\newcommand{\pp}[1]{\left\lVert #1 \right\rVert _p} % p-adisk absolut vÊrdi 2 streger

% Kommandoer
\newcommand{\abs}[1]{\left\lvert #1 \right\rvert} % absolut vÊrdi
\newcommand{\pdiff}[2]{\frac{\partial #1}{\partial #2}} %partielle differentialer
\DeclareMathOperator*{\maks}{maks} 
\newcommand{\aut}{\mathrm{Aut}}

% Ekstra kommandoer Mads
\newcommand{\iprod}[2]{\langle #1 , #2\rangle}
\newcommand{\spor}[1]{\textrm{tr}( #1 )}
\newcommand{\dnorm}[1]{\Vert #1 \Vert}
\newcommand{\enorm}[1]{\left\lvert #1 \right\rvert}
\newcommand{\sset}[1]{\lbrace #1 \rbrace}
\newcommand{\Sset}[1]{\left\{ #1 \right\}}
\newcommand{\msum}{\sum_{\alpha \in \mathbb N_0^n}} 

\theoremstyle{plain}
\newtheorem{lemma}[thm]{Lemma}

\theoremstyle{remark}
\newtheorem*{bemrk}{Remark}

\usepackage[draft,english]{fixme} %in text: \fxnote{..}
\fxsetup{layout=footnote,marginclue}

\makeatletter
\makechapterstyle{burner}{%
  \renewcommand{\chapnamefont}{\normalfont\LARGE\scshape}
  \renewcommand{\printchaptername}{\raggedleft\chapnamefont \@chapapp}
  \renewcommand{\chapnumfont}{\normalfont\Huge}
  \setlength{\chapindent}{\marginparsep}
  \addtolength{\chapindent}{\marginparwidth}
  \addtolength{\chapindent}{-2cm}
  \renewcommand{\printchaptertitle}[1]{%
    \begin{adjustwidth}{}{}%-\chapindent}
      \raggedleft \chaptitlefont ##1\par\nobreak
    \end{adjustwidth}}
}
\makeatother


% Citater
\newsavebox{\citeret}
\newenvironment{citat}[1]{%
  \sbox{\citeret}{--- \textsf{#1}}%
  \begin{flushright}\itshape}%
  {\par \usebox{\citeret}\end{flushright}}

%Functions
\newcommand{\Char}{\text{char}}
\newcommand{\probi}{\text{prob}}

%Algorithms
%\usepackage[section]{algorithm}
%\usepackage{algorithmic}
%\renewcommand{\algorithmicrequire}{\textbf{Input:}}
%\renewcommand{\algorithmicensure}{\textbf{Output:}}
%\usepackage{algorithm2e}

\usepackage[section]{algorithm}
\usepackage{algpseudocode}
\renewcommand{\algorithmicrequire}{\textbf{Input:}}
\renewcommand{\algorithmicensure}{\textbf{Output:}}
\algnotext{EndFor}
\algnotext{EndIf}
\algnotext{EndWhile}

\renewcommand{\thealgorithm}{} % Remove algorithm numbering

\usepackage{caption}

%\newenvironment{game}[1][htb]
%  {\renewcommand{\ALG@name}{Game}% Update algorithm name
%   \begin{algorithm}[#1]%
%  }{\end{algorithm}}
\makeatletter
\addto\captionGame{\renewcommand{\ALG@name}{Game}}
\addto\captionAdversary{\renewcommand{\ALG@name}{Adversary}}
\addto\captionUser{\renewcommand{\ALG@name}{User}}
\addto\captionDistinguisher{\renewcommand{\ALG@name}{Distinguisher}}
\makeatother
%\newcommand{\newalgname}[1]{%
%  \renewcommand*{\ALG@name}{#1}%
%}

\floatstyle{plain}
\newfloat{myalgo}{tbhp}{mya}

\newenvironment{Algorithm}[2][tbh]%
{\begin{myalgo}[#1]
\centering
\begin{minipage}{#2}
\begin{algorithm}[H]}%
{\end{algorithm}
\end{minipage}
\end{myalgo}}




% Length between two adjacent floats
\setlength\floatsep{1\baselineskip plus 3pt minus 2pt} 
%For floats on top and bottom of text only
%(For floats at top - length between float and text below it)
%(For floats at bottom - length between float and text above it)
\setlength\textfloatsep{1\baselineskip plus 3pt minus 2pt} 
% For floats in the middle of text only - length between text above it, and text below it.
\setlength\intextsep{1\baselineskip plus 3pt minus 2 pt}
