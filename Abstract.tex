
\begin{center}
  \large\textbf{Abstract}
\end{center}

We present a novel approach to the topic of cryptographic subversion by introducing the technique of \emph{canonical} cryptographic subversion. We abstract the idea of a cryptographic game in which a notion of subvertability is defined. In the Random Oracle model, we present two canonical cryptographic subversions in the shape of the $\CDH$ and $\DDH$ hardness assumptions. The $\CDH$ subversion is then used to construct a subversion, in the Random Oracle Model, of the $\elGamal$ public-key encryption scheme. In connection with introducing the technique of canonical cryptographic subversion, we also define a definitional framework that unify the theory of cryptographic subversion in a common body of language. A result by Bellare, Paterson and Rogaway is recovered in our new framework that affirms this. 

\cleardoublepage
