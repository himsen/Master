\chapter{Conclusion and Future Work}
\label{cha:conclusionFutureWork}

The ultimate conclusion of this thesis is that unverifiable algorithmic choice can be a significant liability. A subversion attack highlight an attack strategy that is different from conventional attacks: the takeover of a user's software for malicious purposes. The usual protection mechanisms such as testing, auditing or monitoring can not in general ensure detection and security since the attacker can use cryptographic methods to hide its modifications. 

To investigate the state of possible cryptographic subversions, we define a framework for defining cryptographic subversions in a coherent manner. In this framework a technique of canonical cryptographic subversion is defined that is used to subvert the $\elGamal$ public-key encryption scheme. In addition, we frame a previous described cryptographic subversion in out new framework.

There are mainly two interesting directions for future work. First is to describe more previous work in the new framework. Second is to come up with more canonical subversions and show they can be used to subvert lower-level schemes. Interesting opportunities for new canonical subversions are the $\RSA$ and $\LWE$ hardness assumptions. The latter would give a novel method of subverting state-of-the-art lattice based public-key encryption schemes. 

